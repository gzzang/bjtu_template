\chapter{1级标题}\label{cha_one}
\section{2级标题}\label{sec_two}
\subsection{3级标题}\label{subsec_three}
\subsubsection{4级标题}\label{subsubsec_four}

下面展示如何引用章节:“如\bjturef{cha_one}所示”、“如\bjturef{sec_two}所示”、“如\bjturef{subsec_three}所示”、“如\bjturef{subsubsec_four}所示”。

引用著作\bjtucite{Patriksson1994},引用论文\bjtucite{Rosenthal1973},引用中文论文\bjtucite{zhang2020},引用网页\bjtucite{wangye2020}。

下面展示如何引用定理:“如\bjturef{def_1}所示”、“如\bjturef{prop_2}所示”。如需添加新的定理环境,请修改模板文件。
\begin{defn}\label{def_1}
	这是一个定义。
\end{defn}
\begin{prop}\label{prop_2}
	这是一个命题。
\end{prop}

下面展示如何引用公式:“如\bjturef{eq_emc}所示”
\begin{equation}\label{eq_emc}
	E=mc^2
\end{equation}

下面展示如何引用图片:“如\bjturef{fig_label_2}所示”。

\begin{bjtufigure}
\includegraphics[width=\textwidth]{contents/annex/fig1}
	\bicaption[章鱼]{章鱼}{Octopus}
	\label{fig_label_2}
\end{bjtufigure}

下面展示如何引用子图:“如\bjturef{fig_sub_1}所示”。

\begin{bjtufigure}
	\subfigure[章鱼]{\includegraphics[width=0.49\textwidth]{contents/annex/fig1}\label{fig_sub_1}}
	\subfigure[女孩]{\includegraphics[width=0.49\textwidth]{contents/annex/fig2}\label{fig_sub_2}}
	\bicaption[章鱼和女孩]{章鱼和女孩}{Octopus and girl}
	\label{fig_label}
\end{bjtufigure}



下面展示如何引用表格:“如\bjturef{tab_institute}所示”。

\begin{bjtutable}
	\bicaption[研究所]{研究所}{Institute}
	\label{tab_institute}
	\begin{tabularx}{\textwidth}{X<{\centering}X<{\centering}X<{\centering}}
		\toprule
		学校&学院&系所\\
		\midrule
		北京交通大学&交通运输学院&系统科学所\\
		\bottomrule
	\end{tabularx}
\end{bjtutable}

下面展示如何引用跨页表格:“如\bjturef{tab_institute}所示”。

\setlength{\LTleft}{0pt} \setlength{\LTright}{0pt}

{\small
\begin{longtable}{@{\extracolsep{\fill}}cccc@{}}
\bicaption[研究所们]{研究所们}{Institutes}
\label{tab_institutes}\\

\toprule
学校&学校&学院&系所 \\ \midrule
\endfirsthead

\multicolumn{4}{l}{\tablename\ \thetable\ (续)}\\
\toprule
学校&学校&学院&系所 \\ \midrule
\endhead

\bottomrule
\endfoot

\bottomrule
\endlastfoot

北京交通大学&北京交通大学&交通运输学院&系统科学所\\
北京交通大学&北京交通大学&交通运输学院&系统科学所\\
北京交通大学&北京交通大学&交通运输学院&系统科学所\\
北京交通大学&北京交通大学&交通运输学院&系统科学所\\
北京交通大学&北京交通大学&交通运输学院&系统科学所\\
北京交通大学&北京交通大学&交通运输学院&系统科学所\\
北京交通大学&北京交通大学&交通运输学院&系统科学所\\
北京交通大学&北京交通大学&交通运输学院&系统科学所\\
北京交通大学&北京交通大学&交通运输学院&系统科学所\\
北京交通大学&北京交通大学&交通运输学院&系统科学所\\
北京交通大学&北京交通大学&交通运输学院&系统科学所\\
北京交通大学&北京交通大学&交通运输学院&系统科学所\\
北京交通大学&北京交通大学&交通运输学院&系统科学所\\
北京交通大学&北京交通大学&交通运输学院&系统科学所\\
北京交通大学&北京交通大学&交通运输学院&系统科学所\\
北京交通大学&北京交通大学&交通运输学院&系统科学所\\
北京交通大学&北京交通大学&交通运输学院&系统科学所\\
北京交通大学&北京交通大学&交通运输学院&系统科学所\\
北京交通大学&北京交通大学&交通运输学院&系统科学所\\
北京交通大学&北京交通大学&交通运输学院&系统科学所\\
北京交通大学&北京交通大学&交通运输学院&系统科学所\\
北京交通大学&北京交通大学&交通运输学院&系统科学所\\
北京交通大学&北京交通大学&交通运输学院&系统科学所\\
北京交通大学&北京交通大学&交通运输学院&系统科学所\\
北京交通大学&北京交通大学&交通运输学院&系统科学所\\
北京交通大学&北京交通大学&交通运输学院&系统科学所\\
北京交通大学&北京交通大学&交通运输学院&系统科学所\\
北京交通大学&北京交通大学&交通运输学院&系统科学所\\
北京交通大学&北京交通大学&交通运输学院&系统科学所\\
北京交通大学&北京交通大学&交通运输学院&系统科学所\\
北京交通大学&北京交通大学&交通运输学院&系统科学所\\
北京交通大学&北京交通大学&交通运输学院&系统科学所\\
北京交通大学&北京交通大学&交通运输学院&系统科学所\\
北京交通大学&北京交通大学&交通运输学院&系统科学所\\
北京交通大学&北京交通大学&交通运输学院&系统科学所\\
北京交通大学&北京交通大学&交通运输学院&系统科学所\\
北京交通大学&北京交通大学&交通运输学院&系统科学所\\
北京交通大学&北京交通大学&交通运输学院&系统科学所\\
北京交通大学&北京交通大学&交通运输学院&系统科学所\\
北京交通大学&北京交通大学&交通运输学院&系统科学所\\
北京交通大学&北京交通大学&交通运输学院&系统科学所\\
北京交通大学&北京交通大学&交通运输学院&系统科学所\\
\end{longtable}
}

下面展示如何引用算法:“如\bjturef{alg_ptn}所示”。

\begin{algorithm}[htbp] 
	\caption{针对个体异质交通网络均衡问题的最佳反应算法}
	\label{alg_ptn} 
	\begin{algorithmic}[1]
	 \Require
	 博弈问题; 
	 \Ensure 
	 均衡策略组合$\widehat{\bm{s}}$;\\
	 \textbf{初始化:}在策略组合集合中,随机选择一个策略组合$\widehat{\bm{s}}$,作为初始策略组合。即每位出行者等概率选择一种策略。
	 \For{出行者${\ell}\in\mathcal{L}$(生成$\mathcal{L}$的一个随机排列进行循环)}
	 \For{策略${r\in\mathcal{S}_{\ell}}$(生成$\mathcal{S}_{\ell}$的一个随机排列进行循环)}
	 \If{$f_{{c}_{\ell}}\left(f_{\bm{\phi}}\left(\widehat{\bm{s}},\ell,{r}\right)\right)<f_{{c}_{\ell}}\left(\widehat{\bm{s}}\right)$}
	 \State $\widehat{\bm{s}}\leftarrow{}f_{\bm{\phi}}\left(\widehat{\bm{s}},\ell,{r}\right)$
	 \State 转步骤2
	 \EndIf
	 \EndFor
	 \EndFor
	 \State \Return 策略组合$\widehat{\bm{s}}$
	\end{algorithmic} 
   \end{algorithm}

下面测试中English混\hspace{0em}[1]排。

\chapter{1级标题}
\section{2级标题}
\subsection{3级标题}
\subsubsection{4级标题}
\subsubsection{4级标题}
\subsection{3级标题}
\subsubsection{4级标题}
\subsubsection{4级标题}
\section{2级标题}
\subsection{3级标题}
\subsubsection{4级标题}
\subsubsection{4级标题}
\subsection{3级标题}
\subsubsection{4级标题}
\subsubsection{4级标题}
